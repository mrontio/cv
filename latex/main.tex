% Following CV tutorial https://latex-tutorial.com/cv-latex-guide/
\documentclass[11pt]{article}

\usepackage[english]{babel}
\usepackage[utf8]{inputenc}
\usepackage{blindtext}

\usepackage{textcomp}
\usepackage{caption}
\usepackage{tgpagella}
\usepackage{graphicx}
\usepackage{geometry}
\geometry{
    a4paper,
    left=20mm,
    %right=
    bottom=10mm,
    top=15mm
}

\usepackage{titlesec}
\titlespacing*{\section}
{0pt}{\baselineskip}{\baselineskip} % left, before, after

\pagestyle{empty}

\usepackage{enumitem}

\usepackage{sectsty}
\sectionfont {
    \raggedright
    \large
    \fontfamily{bch}\selectfont
    \sectionrule{0pt}{0pt}{-7pt}{1pt} % insert a thin rule
}

\usepackage{hyperref}
\hypersetup{
    colorlinks=true,
    filecolor=magenta,
    urlcolor=blue,
    pdftitle={CV - Michail Rontionov},
}


% Macros
% Spacebox length is the length of the text {1..9}
\newlength{\spacebox}
\settowidth{\spacebox}{123456789}
\newcommand{\sepspace}{\vspace*{1em}}

\newcommand{\name}[1]{
    \Huge
    \fontfamily{phv}\selectfont
    \begin{center} \textbf{#1} \end{center}\par
    \normalsize\normalfont
}

\newcommand{\desc}[1]{
    \normalsize
    \fontfamily{phv}\selectfont
    \begin{center} \textsl{#1} \end{center}\par
    \normalsize\normalfont
}

\newcommand{\info}[2]{
    \noindent\hangindent=2em\hangafter=0
    \parbox{\spacebox}{%
    \textsl{#1}}
    #2 \par
}

\newcommand{\skill}[2]{
    % set indent for personal info
    \noindent\begin{minipage}{0.3\textwidth}
    \textbf{#1}
    \end{minipage}
    \hfill
    \begin{minipage}{0.7\textwidth}
    #2
    \end{minipage}
    \vspace*{0.5em}
    \par
    %\parbox{3\spacebox}{
    %   \textbf{#1}
    %}
    %#2 \par
}

\newcommand{\education}[5]{
    \noindent \textbf{#1}
    \hfill
    \framebox {
    \parbox{6em} {
        \centering\textbf{#2 -- #3}
    }}
    \par
    \noindent \textit{#4} \par
    \hangindent=2em\hangafter=0 \small #5
    \normalsize \par
    \vspace*{0.5em}
}

\newcommand{\work}[5] {
    \noindent \textit{#1}
    \hfill
    \framebox {
    \parbox{6em} {
        \centering\textbf{#2 -- #3}
    }}
    \par
    \noindent \textbf{#4} \par
    \hangindent=2em\hangafter=0 \small #5
    \normalsize \par
}

\newcommand{\volunteering}[4] {
    \noindent \textit{#1}
    \hfill
    \framebox {
    \parbox{6em} {
        \centering\textbf{#2}
    }}
    \par
    \noindent \textbf {#3} \par
    \noindent\hangindent=2em\hangafter=0 \small #4
    \normalsize \par
}

\newcommand{\summary}[1]{  \large % font size
  \fontfamily{phv}\selectfont % font family
% print motto centered and slanted
  \begin{center} \textsl{#1}\end{center}\par
% back to normal size and font
  \normalsize \normalfont}

%%% Local Variables:
%%% mode: LaTeX
%%% TeX-master: "main"
%%% End:


\begin{document}
    \name{Michail Rontionov}

    \info{Email}{\href{mailto:mrontionov@mront.io}{mrontionov@mront.io}}
    \info{Phone}{\href{tel:+447895682664}{+44 7895 682664}}
    \info{GitHub}{\href{https://github.com/mrontio}{github.com/mrontio}}
    \info{LinkedIn}{\href{https://www.linkedin.com/in/michail-rontionov-64810b183/}{Michail Rontionov}}
    \info{Location}{Southampton, UK}

    \section*{Personal Statement}
    \summary{A PhD student interested in novel forms of computing, lead by curiosity and people.}

    \section*{Education}
    \education{PhD in Artificial Intelligence, Compter Science and Electronics}{2023}{2027}{University of Southampton}{
      \noindent Integrated PhD as part of MINDS CDT.
      \begin{itemize}
      \item Currently in month 12 of 36 of PhD with working dissertation \\``Neuromorphic Computing: Representations, Compilation and Digital Architectures.''
      \item Supervised by Prof. David Thomas and Dr. Ruomeng Huang.
      \item Self-identified PhD topic based on interests in electronics, computer science and AI.
      \item Attended multiple neuromorphic conferences and found collaborators, first paper to be submitted to ISCAS in mid-October.
      \item Currently creating a compiler from spiking neural networks to FPGAs, creating a dedicated accelerator, and learning Agda to create a type-safe language for neuromorphic computing.
      \end{itemize}
    }
    \education{MSci in Computer Science with a Year in Industry}{2018}{2023}{Swansea University}{
      \noindent Integrated Master's with First Class Honours.
      \begin{itemize}
      \item MSci Dissertation: Higher-order framework for the PYNQ FPGA Ecosystem, supervised by Dr. Cécilia Pradic, to enable using hardware objects as higher-order primitives (map, reduce, etc.)
      \item BSc Dissertation: Implementing Single-Event Upset Fault Tolerance onto a RISC-V Core, supervised by Dr. Shane Fleming, which allowed a RISC-V core to be used in satellites.
      \end{itemize}
    }




    \section*{Work Experience}
    \work{HPC Engineer Intern}{2020}{2021}{Intel Corporation}{
      \begin{itemize}
      \item Research internship role involving finding unique solutions to niche but expensive problems, ultimately made me interested in a PhD.
      \item Profiled financial customer applications with VTune and achieved 1.5x to 4x imporevements in memory speeds and throughput purely with AVX512, C intrinics and proprietary knowledge.
      \item Authored a tool to map logical CPU cores to physical cores based on debug statistics for cache allocation improvements, reduced average cache transfer latencies by 25\%.
      \item Maintained the Swindon lab for the HPC working group, which included installing new systems, upgrading existing ones, and fixing issues as they arose.
      \end{itemize}
    }
    \newpage
    \work{Computer Science Teaching Assistant}{2019}{2023}{Swansea University}{
      Marked first to third year labs as a side job alongside the undergraduate degree. Helped students with questions relating to Java, Haskell, Prolog, and computer science theory.
    }
    \work{I.T. Manager}{2017}{2018}{Melody Real Estate Management}{
    Part-time job during Secondary School. Sole I.T. employee for a team of 6 people; required understanding and acting on employee's concerns, fixing day-to-day I.T. issues, and maintaining systems.
    }

    \section*{Skills}
    \skill{Languages}{C, C++, Python, Java, Verilog, Haskell, Scala, SpinalHDL.}
    \skill{Hardware}{PYNQ Z2, Raspberry Pi \{4b, 5\}, Arduino, Dell EMC, servers, RISC-V, x86.}
    \skill{Software}{Vitis, Vitis HLS, Vivado, Linux, NixOS, Emacs, \LaTeX.}
    \skill{Soft Skills}{Team work, leadership, teaching, working with children, self-management, problem solving, time management.}
    \skill{Physical Skills}{Computer Maintenance, Electronics Soldering, Bicycle Repair.}

    \section*{Volunteering}
    \volunteering{Ambassador Volunteer}{2018 --- 2020}{Technocamps}{
    Contributed time to organisation providing after-school science activities to children aged 10 to 16. Required understanding of children's intuition and develop explanations which worked for each individual, and managed the environment for a fun and engaging learning class.}
    \volunteering{Swansea University Ecological Volunteering}{2021 --- 2022}{Swansea University}{
    Cleaned up beaches and uprooted trees in Swansea's ecologically sensitive habitats to promote local flora and fauna.}
    \volunteering{General Volunteering}{2018 --- 2019}{Discovery Volunteering}{Numerous volunteering opportunities, including running events for people in need of help, preparing hygiene kits for homeless and park cleanings.}

    \section*{Interests}
    \begin{itemize}
    \item Systems --- I enjoy building homelab systems on the cheap.
    \item Cycling --- Choice of sport since I was a teenager, currently into bike packing.
    \item Hiking --- Camping and hiking is what I always aim to do more of.
    \item Reading --- Currently : The Joy of Abstraction: An Exploration of Math, Category Theory and Life by Eugenia Cheng.
    \end{itemize}

\end{document}

%%% Local Variables:
%%% mode: LaTeX
%%% TeX-master: t
%%% End:
